%!TEX root = ../ENTRUST_TR.tex

%This section will explain what are we going to do, which is combining two technologies ActivFORMS + Prism.
%\simos{References needed}
The increasing complexity of software systems combined with the need to develop more versatile, resilient, and reliable systems instigated the vision of self-managing systems, also known as autonomous computing\cite{Kephart2003:Comp, Ganek2003:IBM}. In this regard, self-managing systems are expected not only to operate with minimal administrator intervention throughout their lifetime, but also to adapt while providing service in response to environment changes, evolving requirements and unexpected failures. As a result, the software engineering research community devoted much effort in developing frameworks, methodologies and architectures that support the engineering of systems enhanced with self-* capabilities (e.g., self-adaptive, self-managing, self-healing, self-optimising);~see \cite{Salehie2009:TAAS, Huebscher2008:ACM} for more information. 

Despite the research progress achieved since the advent of autonomous computing, the engineering of trustworthy self-managing systems remains a major research challenge. This barrier constrains the applicability of self-managing systems in safety-critical and business-critical applications, as for example, in healthcare, e-commerce, defence and finance. Systems deployed in these application areas should work dependably and must be characterised by high-integrity runtime operation, as defined by NIST~\cite{NIST}: ``High integrity software is software that must be trusted to work dependably in some critical function, and whose failure to do so may have catastrophic results, such as serious injury, loss of life or property, business failure or breach of security."

The software engineering community classified the provision of evidence, i.e., assurances, that a system operates dependably throughout its lifetime among the most important research objectives for self-managing systems~\cite{Cheng2009:Dagstuhl}. In fact, assurances was amid the research threads highlighted in the most recent roadmaps for self-adaptive systems~\cite{Lemos2013:Dagstuhl, Lemos2014:Dagstuhl}. We define assurances as \textit{``the provision of evidence that the system satisfies its stated functional and non-functional requirements during its operation in the presence of self- adaptation"}~\cite{Lemos2014:Dagstuhl}.

In this work, we extend the current practices on the provision of assurances for self-managing systems. To this end, we introduce the first tool-supported methodology for assuring the safety of the closed control loops of self-managing systems using established industry practices. Our framework for the ENgineering of TRUstworthy Self-managing sysTems (\approach) integrates:
\squishlist
	\item an extended version of our approach to developing formally verified monitor-analysis-planning-execution (MAPE) autonomic control loops~\cite{Iftikhar2014:SEAMS};
	\item our runtime quantitative verification technique for the analysis stage of these loops~\cite{Calinescu2012:CACM};
	\item an industry-adopted approach to arguing security~\cite{Weinstock2007} based on the widely used Goal Structuring Notation (GSN) for safety arguments~\cite{Kelly2004:DSN}.
\squishend

The report is organised as follows.  Section~\ref{sec:example} describes the self-adaptive unmanned marine vehicle embedded system used for illustrating and evaluating \approach. Sections~\ref{sec:activForms} and~\ref{sec:rqv} introduces the approaches underpinning \approach, while Section~\ref{sec:implementation} presents the realisation of \approach\ for the implementation of the embedded system described in Section~\ref{sec:example}..

%The theoretical background underpinning \approach\ is introduced in Section~\ref{sec:preliminaries}, while Section~\ref{sec:entrust} describes the \approach\ methodology, and 
